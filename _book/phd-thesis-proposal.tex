% Options for packages loaded elsewhere
\PassOptionsToPackage{unicode}{hyperref}
\PassOptionsToPackage{hyphens}{url}
%
\documentclass[
]{book}
\usepackage{lmodern}
\usepackage{amssymb,amsmath}
\usepackage{ifxetex,ifluatex}
\ifnum 0\ifxetex 1\fi\ifluatex 1\fi=0 % if pdftex
  \usepackage[T1]{fontenc}
  \usepackage[utf8]{inputenc}
  \usepackage{textcomp} % provide euro and other symbols
\else % if luatex or xetex
  \usepackage{unicode-math}
  \defaultfontfeatures{Scale=MatchLowercase}
  \defaultfontfeatures[\rmfamily]{Ligatures=TeX,Scale=1}
  \setmainfont[]{Arial}
  \setmathfont[]{Fira Math Regular}
\fi
% Use upquote if available, for straight quotes in verbatim environments
\IfFileExists{upquote.sty}{\usepackage{upquote}}{}
\IfFileExists{microtype.sty}{% use microtype if available
  \usepackage[]{microtype}
  \UseMicrotypeSet[protrusion]{basicmath} % disable protrusion for tt fonts
}{}
\makeatletter
\@ifundefined{KOMAClassName}{% if non-KOMA class
  \IfFileExists{parskip.sty}{%
    \usepackage{parskip}
  }{% else
    \setlength{\parindent}{0pt}
    \setlength{\parskip}{6pt plus 2pt minus 1pt}}
}{% if KOMA class
  \KOMAoptions{parskip=half}}
\makeatother
\usepackage{xcolor}
\IfFileExists{xurl.sty}{\usepackage{xurl}}{} % add URL line breaks if available
\IfFileExists{bookmark.sty}{\usepackage{bookmark}}{\usepackage{hyperref}}
\hypersetup{
  pdftitle={Evan's PhD thesis proposal},
  pdfauthor={Evan C Mascitti},
  hidelinks,
  pdfcreator={LaTeX via pandoc}}
\urlstyle{same} % disable monospaced font for URLs
\usepackage{longtable,booktabs}
% Correct order of tables after \paragraph or \subparagraph
\usepackage{etoolbox}
\makeatletter
\patchcmd\longtable{\par}{\if@noskipsec\mbox{}\fi\par}{}{}
\makeatother
% Allow footnotes in longtable head/foot
\IfFileExists{footnotehyper.sty}{\usepackage{footnotehyper}}{\usepackage{footnote}}
\makesavenoteenv{longtable}
\usepackage{graphicx}
\makeatletter
\def\maxwidth{\ifdim\Gin@nat@width>\linewidth\linewidth\else\Gin@nat@width\fi}
\def\maxheight{\ifdim\Gin@nat@height>\textheight\textheight\else\Gin@nat@height\fi}
\makeatother
% Scale images if necessary, so that they will not overflow the page
% margins by default, and it is still possible to overwrite the defaults
% using explicit options in \includegraphics[width, height, ...]{}
\setkeys{Gin}{width=\maxwidth,height=\maxheight,keepaspectratio}
% Set default figure placement to htbp
\makeatletter
\def\fps@figure{htbp}
\makeatother
\setlength{\emergencystretch}{3em} % prevent overfull lines
\providecommand{\tightlist}{%
  \setlength{\itemsep}{0pt}\setlength{\parskip}{0pt}}
\setcounter{secnumdepth}{5}
\usepackage{booktabs}
\ifluatex
  \usepackage{selnolig}  % disable illegal ligatures
\fi
\usepackage[]{natbib}
\bibliographystyle{apalike}

\title{Evan's PhD thesis proposal}
\author{Evan C Mascitti}
\date{last updated 2020-10-25}

\begin{document}
\maketitle

{
\setcounter{tocdepth}{1}
\tableofcontents
}
\hypertarget{overview}{%
\chapter{Overview}\label{overview}}

Baseball fields are built from soil. Therefore, soil behavior determines the safety and function of the field. The purpose of this dissertation is to study how soil behavior relates to baseball infields.

The production of artificial soils has received much study. However, there are
no published scientific experiments on this topic as it pertains to baseball and
softball fields.

This proposal is organized into three sections:

\protect\hyperlink{purpose-objectives}{\textbf{Purpose and objectives}}: A broad description of the scope and aims of the project

\protect\hyperlink{definition-of-terms}{\textbf{Definition of terms}}: Explicit meanings of verbiage used this document (these may vary across scientific fields)

\protect\hyperlink{lit-review}{\textbf{Review of literature}}: A summary of scientific research pertaining to this topic

\protect\hyperlink{proposed-experiments}{\textbf{Proposed experiments}}: The planned course of laboratory research, organized as anticipated publication units.

\hypertarget{purpose-objectives}{%
\chapter{Purpose and objectives}\label{purpose-objectives}}

The purpose of my thesis is to create a new way to think about the soils used on baseball and softball infields.

\textbf{My objective is to answer these open-ended questions:}

\begin{enumerate}
\def\labelenumi{\arabic{enumi}.}
\tightlist
\item
  How can infield soil performance be quantified without an experienced grounds manager's expertise?
\item
  How can an infield soil mixture be designed to achieve a given set of desired properties?
\item
  What simple lab tests can be used to predict the performance of the mix?
\end{enumerate}

\textbf{The anticipated outcomes and deliverables from this project are:}

\begin{enumerate}
\def\labelenumi{\arabic{enumi}.}
\tightlist
\item
  A general framework for understanding the behavior of sand-clay mixes in a quantitative way
\item
  Specific recommendations for the ratios which common clayey soils should be mixed with sand for optimal performance at any maintenance level
\item
  Suggestions for how to beneficiate poorly-performing clay soils with other kinds of clays to maximize the locally available materials by ``spiking'' the local clay with a small percentage of imported solum
\end{enumerate}

\hypertarget{definition-of-terms}{%
\chapter{Definition of terms}\label{definition-of-terms}}

\textbf{clay}: a soil material which can be permanently reshaped or molded when moist and which retains develops high strength when dry

\textbf{clay-size}: an individual mineral grain which settles in through a column of fluid at the same rate as a spherical particle of 2 μm

\textbf{sand}: a soil material which is predominantly composed of mineral particles having sieve diameters between 53 and 2000 μm

\textbf{packing fraction}: the volume fraction occupied by solids, normalized to the total soil volume (1-void fraction); \(V_{s}\)

\hypertarget{lit-review}{%
\chapter{Review of literature}\label{lit-review}}

This literature review is organized into four sections to address the following topics:

\textbf{\ref{infield-performance}} \protect\hyperlink{infield-performance}{\textbf{Function and performance of baseball/softball infields}}

\textbf{\ref{soil-behavior-fundamentals}} \protect\hyperlink{soil-behavior-fundamentals}{\textbf{Soil behavior terminology and measures}}

\textbf{\ref{artificial-soil-mixtures}} \protect\hyperlink{artificial-soil-mixtures}{\textbf{Design and properties of artificial soil mixtures}}

\textbf{\ref{lab-methods-review}} \protect\hyperlink{lab-methods-review}{\textbf{Laboratory test methods pertinent to the research questions}}

\hypertarget{infield-performance}{%
\section{Function and performance of baseball infields}\label{infield-performance}}

Athletes engage with the playing surface in two ways: directly (by running, pivoting, and sliding) and indirectly (by fielding batted balls).

During play, forces are imposed to the infield skin by athletes feet and bodies, and by the ball. This section outlines the goals of a baseball grounds manager and prior scientific research (scant as it is) on this topic.

\hypertarget{qualitative-description-of-the-importance-of-the-infield}{%
\subsection{Qualitative description of the importance of the infield}\label{qualitative-description-of-the-importance-of-the-infield}}

A full-size baseball field occupies 1 ha. About 74\% of this total area is surfaced with natural turfgrass or synthetic turf. An additional \textasciitilde{} 16\% is occupied by the warning track, which is designed to alert players that they are nearing the wall. Only about 8\% of the total playing surface is occupied by the infield skin. However, the majority of the game is played on this area, and at any given moment the all the offensive players, the four infielders, and the pitcher and catcher are standing on the infield skin. Because of its importance to athlete safety and performance, the infield skin consumes a majority of labor and material expenditures (see Figure \ref{fig:labor-per-area} ).

\textbackslash begin\{figure\}
\includegraphics[width=0.4\linewidth]{images/player-locations} \includegraphics[width=0.4\linewidth]{phd-thesis-proposal_files/figure-latex/labor-per-area-2} \textbackslash caption\{A. Locations of players during a professional baseball game; note the paucity of players on turf (blue arrows) compared to bare soil/skin areas (red arrows). B. The infield skin consumes the majority of labor input despite comprising only \textasciitilde8\% of the playing surface.\}\label{fig:labor-per-area}
\textbackslash end\{figure\}

\hypertarget{research-on-infield-surface-performance}{%
\subsection{Research on infield surface performance}\label{research-on-infield-surface-performance}}

\citet{Goodall2005} is the only published account of research on infield soil mixtures. The authors installed several soils which were commercially available within their region.

\citet{Brosnan2008a} surveyed the surface conditions of the infield skin on extant playing fields at three maintenance levels. Particle size analyses were performed on soil sampled from each infield skin. The USDA soil texture of those samples is plotted in \ref{fig:brosnan-survey-usda}. These soils were sampled from the upper 13 mm and contained large granules of calcined clay infield conditioner; therefore, the texture measured with this method is coarser than the ``true'' texture of the base soil.

Additionally, Brosnan et al.~published research on the infield skin's role in athlete-to-surface interactions \citeyearpar{Brosnan2009} and ball-to-surface interactions \citeyearpar{Brosnan2011}. Bulk density (\(\Large{\rho}\small{_{bulk}}\)) was shown to influence surface properties, although the range of bulk densities tested (1.2-1.8 Mg m\textsuperscript{-3}) was beneath values typically encountered on infield skins (author's personal observation; data not shown).

However, the work of Brosnan et al. \citetext{\citeyear{Brosnan2009}; \citeyear{Brosnan2011}} was performed on a single soil material and focused on construction and maintenance practices rather than mix design.

\hypertarget{use-of-artificial-soil-mixtures-on-baseball-infields}{%
\subsection{Use of artificial soil mixtures on baseball infields}\label{use-of-artificial-soil-mixtures-on-baseball-infields}}

Baseball was first played in the early 19th century, but the definitive origins of the game are likely lost to history \citep{Walker1994}. The earliest recorded attempt to alter the physical properties of an infield soil were by Harry Wright in 1875. Wright and his contemporaries incorporated various materials into their infield soils to enhance stability, firmness, or drainage of the playing surface. Amendments included organic debris (straw, ashes) and and inorganic materials (sand, lime, cinders) \citep{Morris2007}.

Infield soil mixes were produced off-site and imported beginning in the 1960s ?Zwasksa?.

\hypertarget{soil-behavior-fundamentals}{%
\section{Soil behavior terminology and measures}\label{soil-behavior-fundamentals}}

Toughness is really the most defining feature of clay soil.

\hypertarget{artificial-soil-mixtures}{%
\section{Design and properties of artificial soil mixtures}\label{artificial-soil-mixtures}}

Natural soil materials are excavated and deliberately blended for many uses. Potting mixes and green roof media are the largest means of production by volume (reference ??). The purpose of blending multiple soil materials is to create a product having properties not exhibited by any naturally occurring soil material. Even if the desired properties \emph{could} be found in a single naturally occurring material, the properties of natural soils are subject to spatial variation. In such a scenario, the properties of the mixture can be held constant simply by adjusting the ratio of the components.

The Atterberg limits and unconfined compression testing are the primary means which have been used to charazcterize soil mixtures. These mixtures may contain two components(sand and clay), or three components (sand and two separate types of clay soil)

\hypertarget{lab-methods-review}{%
\section{Laboratory methods for evaluating soil behavior and physical properties}\label{lab-methods-review}}

\hypertarget{particle-size-analysis}{%
\subsection{Particle size analysis}\label{particle-size-analysis}}

\hypertarget{compaction-tests}{%
\subsection{Compaction tests}\label{compaction-tests}}

\hypertarget{compression-and-shear-strength-tests}{%
\subsection{Compression and shear strength tests}\label{compression-and-shear-strength-tests}}

\hypertarget{atterberg-limits}{%
\subsection{Atterberg limits}\label{atterberg-limits}}

\hypertarget{origins-of-the-test-methods}{%
\subsubsection{Origins of the test methods}\label{origins-of-the-test-methods}}

A common definition of plasticity is the tendency of a material to deform under an applied load, without fracturing into multiple pieces, and to retain its new shape when the load is removed \citep{Andrade2011} . The earliest test methods for soil plasticity were developed by Atterberg \citeyearpar{Atterberg1911}, \citeyearpar{Atterberg1974}. Atterberg noted that although plasticity is easy to observe, the phenomenon does not lend itself to simple measurement. He showed that the consistency of the soil was determined by its water content, and he reasoned that soil was plastic only within a finite range of water content, which differed for every soil. The upper water content boundary was defined as the flow limit, at which two batches of soil paste flowed together when jarred, and the lower water content boundary was defined as the rolling limit, at which the soil could no longer be rolled into thin threads. Atterberg called the difference between these two characteristic water contents the ``plasticity number'' (now known as the plasticity index or PI). He defined five other thresholds of soil behavior, though the others are not commonly used today.

Atterberg deemed the plasticity number as the most reliable means of measuring plasticity, though upon reading his work one senses a reluctance to accept this simple number as a fully adequate measure of such a complex phenomenon. He pointed out that his `plasticity number' provides no information amount of plastic strain incurred during deformation, nor does it enumerate the stress required to impart the deformation. He considered the effort needed to deform the soil a separate property, which translates to English as `viscosity;' this property was later termed `toughness' by \citet{Casagrande1932}.

The value of Atterberg's work has never been fully realized in his own discipline of soil science. However, the utilty of his test methods was clear to those in the then-novel field of geotechnical engineering. \citet{Terzaghi1926} and \citet{Casagrande1932} modified and standardized the Atterberg limit tests, leading to their widespread adoption for the design and control of earthworks projects. \citet{Terzaghi1926} acknowledged the arbitrary nature of the tests, yet emphasized their value as a preliminary soil classification tool. He scorned qualitative definitions of soil behavior and favored the use of quantitative descriptions:

\begin{quote}
``Every engineer should develop the habit of expressing the plasticity and grain-size characteristics of soils by numerical values rather than adjectives\ldots..the degree of plasticity should be indicated by the estimated value of the plasticity index and not by the words `trace of plasticity' or `highly plastic.'\,''
\end{quote}

\citet{Terzaghi1996}

\hypertarget{mechanics-of-the-liquid-limit-test}{%
\subsubsection{Mechanics of the liquid limit test}\label{mechanics-of-the-liquid-limit-test}}

The liquid limit test described by \citet{Atterberg1911} required the soil to be agitated manually, introducing an unacceptable degree of operator dependence. The test method was first standardized by \citet{Casagrande1932}. The Casagrande method uses a brass cup fixed to a rotating cam, which agitates the soil paste by dropping the cup against a hard rubber base of specified height. A recent survey by \citet{Haigh2016} supports Casagrande's warning that the stiffness of the rubber base can significantly affect the results obtained with this method in addition to the device's geometry and operator technique,.

An alternative test device for liquid limit determination was adopted by BS \_\_\_\_\_\_ (reference from Holtz et al.~?? )

\hypertarget{mechanics-of-the-plastic-limit-test}{%
\subsubsection{Mechanics of the plastic limit test}\label{mechanics-of-the-plastic-limit-test}}

Current protocols for the plastic limit include ASTM D4318, AASHTO T 90, and B.S. 1377. In the test, a soil is first wetted to the liquid limit to encourage full saturation. After performing the liquid limit test, a \textasciitilde10 g sample of soil is gradually dried by gentle blow drying and re-molding with the operator's fingers. Once the soil can be molded without sticking to the operator's skin, the soil is rolled into a thread of 3 mm diameter, broken apart, and pressed into a new lump. This process is repeated until the soil crumbles when the rolling action is applied.

\citet{Terzaghi1926} introduced the use of a fixed thread diameter. Recently, the significance of the thread diameter has been questioned \citep{Barnes2013}. The stability of the soil thread is related to the maximum particle diameter and the rolling technique, and (Barnes, 2013) argued that emphasis should be shifted away from a specific thread diameter and toward observation of the thread during the test.
Efforts have been made to improve the plastic limit test. The following section describes alternative test methods developed to improve upon the original thread-rolling method.

Most attempts to improve the plastic limit test have focused on mechanizing the thread-rolling procedure. Test operators utilize different combinations of force, speed, and displacement when rolling the soil. Collectively these variables were termed `rolling path' by \citep{Barnes2013}. \citet{Bobrowski1992} described a simple apparatus to aid the operator in producing a thread of precisely 3.2 mm. The device consists of a flat plexiglass plate which is used to roll the thread, rather than the operator's hand. (Bobrowski and Griekspoor, 1992) also state that paper should be affixed to the base of the device to prevent the thread from sliding and to expedite the drying process. Use of this device is allowed, but not mandated, in the current version of ASTM D4318. This device has been criticized by (Barnes, 2013), who cited the data of (Rashid et al., 2008) to assert the rolling device produces excessively rapid drying and eliminates the soil thread from the view of the operator.

A fully mechanized thread rolling apparatus was developed by (Temyingyong et al., 2002). Their device used two acrylic plates similar to (Bobrowski and Griekspoor, 1992) and added a DC motor to apply the rolling action. The DC voltage was adjusted to control the rolling speed, and the downward force was altered by the addition of weights to the upper plate. They found that the initial diameter of the soil mass explained a larger amount of variation in the test result than did factors which might be ascribed to the subjective manual method (speed and pressure). The device still appears to be a significant improvement over the hand-rolling method; unfortunately, the device is not commercially available and its use has not been adopted by governing bodies.
Barnes (2009) introduced a novel thread-rolling apparatus which allows precise control of the load applied to the soil thread. The device comprises two stainless steel plates: a fixed base and an upper loading plate which is manually oscillated. The load is adjusted by sliding a weight ballast along the side of the device opposite the handle. The further the ballast weight is from a pivot point, the lesser the load on the soil thread. The device is still operated by hand and a constant rate of rolling must be maintained through careful operation. A thin smear of petrolatum is used on the stainless steel plates to encourage extrusion of the soil thread. A number of other useful properties have been developed with this device, as described in the section of this review on soil toughness.
Moreno-Marato and Alonso-Azcàrate described a plastic limit test in a soil thread is bent rather than rolled. The soil is wetted to a moldable consistency and flattened to \textasciitilde3 mm. A special slicing tool is used to create a rectangular prism of soil having precise dimensions of 3 mm x 3mm x 50 mm. The specimen is then rounded into a cylindrical thread using the same tool. The thread is carefully bent about its center, which is anchored around a stainless steel cylinder. When the thread begins to crack, a caliper is used to measure the distance between the two ends of the thread. The test is repeated for at least two other water contents and the water content of the threads is plotted against the displacement with segmented regression. The shallower segment is extrapolated to zero displacement and this water content is taken as the plastic limit.

Moreno-Marato and Alonso-Azcàrate also described a faster version of their original thread-bending test. In this version only a single thread is prepared and its displacement and water content are extrapolated to zero displacement using an empirical equation. This test meets the original requirements of any plastic limit test which could replace the current method, namely:

\begin{enumerate}
\def\labelenumi{\arabic{enumi}.}
\tightlist
\item
  Rapid
\item
  repeatable
\item
  Operator-independent
\end{enumerate}

\hypertarget{attempts-to-improve-atterberg-limit-methods}{%
\subsubsection{Attempts to improve Atterberg limit methods}\label{attempts-to-improve-atterberg-limit-methods}}

Many soil tests have been developed with the goal of supplanting the Atterberg limits. However, due to the success and the abundance of data which has accumulated using these test methods, they are unlikely to be abandoned.

Most criticism of the Atterberg limit tests center around operator tehcnique.

\hypertarget{toughness-tests}{%
\subsection{Toughness tests}\label{toughness-tests}}

\hypertarget{proposed-experiments}{%
\chapter{Proposed experiments and papers}\label{proposed-experiments}}

\hypertarget{chapter-1-a-novel-method-for-measuring-the-performance-of-baseball-and-softball-infield-skin-soils}{%
\section{Chapter 1: A novel method for measuring the performance of baseball and softball infield skin soils}\label{chapter-1-a-novel-method-for-measuring-the-performance-of-baseball-and-softball-infield-skin-soils}}

\hypertarget{chapter-2-toughness-of-clay-soil-near-the-plastic-limit-using-unconfined-compression-tests}{%
\section{Chapter 2: Toughness of clay soil near the plastic limit using unconfined compression tests}\label{chapter-2-toughness-of-clay-soil-near-the-plastic-limit-using-unconfined-compression-tests}}

\hypertarget{chapter-3-a-critical-appraisal-of-particle-size-analysis-as-a-proxy-for-soil-behavior}{%
\section{Chapter 3: A critical appraisal of particle size analysis as a proxy for soil behavior}\label{chapter-3-a-critical-appraisal-of-particle-size-analysis-as-a-proxy-for-soil-behavior}}

\hypertarget{chapter-4-a-rational-theory-of-mix-design-for-blended-soils-used-on-baseball-and-softball-infields}{%
\section{Chapter 4: A rational theory of mix design for blended soils used on baseball and softball infields}\label{chapter-4-a-rational-theory-of-mix-design-for-blended-soils-used-on-baseball-and-softball-infields}}

\hypertarget{references}{%
\chapter{References}\label{references}}

  \bibliography{book.bib,packages.bib,library.bib}

\end{document}
